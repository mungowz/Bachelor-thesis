%\setcounter{page}{1}
\def\baselinestretch{1}
\chapter*{Capitolo 1} \label{cap1}

\section*{Introduzione}
\def\baselinestretch{1.66}
\noindent In botanica, l’impollinazione è quel processo che consiste nel trasporto dei pollini dalla parte maschile e quella femminile dell’apparato riproduttivo delle piante. Grazie ad agenti atmosferici e sopratutto al lavoro incessante degli insetti impollinatori, soprattutto le api, il polline viene trasportato da una pianta all’altra rendendo possibile la fecondazione di un’essenza vegetale della stessa specie e la conseguente produzione di semi e frutti. Da un rapporto dell’Unione Internazionale per la Conservazione della Natura (LUCN) risulta che il 10\% delle specie selvatiche di api (apis mellifera) sarebbe in via di estinzione e un altro 5\% sarebbe a rischio. Una delle principali cause sono i pesticidi, i quali influenzano l’apprendimento, la capacità riproduttiva, i comportamenti sociali di questi insetti e l'orientamento. \newline
\noindent Lo scopo della presente tesi è quello di illustrare la progettazione di un software che visualizza come le molecole di specifici pesticidi si dispongono, in maniera spaziale, quando sono legate ai recettori delle api, questo processo viene definito \textbf{docking molecolare}, e successivamente il software estrae i legami che si vengono a formare. 

\begin{center}
\includegraphics{img/apis mellifera.png}
\footnotesize \centerline {Esemplare adulto di Apis mellifera}
\end{center}


\section*{1.1 Docking Molecolare}
\def\baselinestretch{1.66}
\noindent Il \textbf{docking molecolare} è un metodo che generalmente consente di predire l’orientamento e le interazioni di una molecola all’interno di una proteina o acido nucleico. I software di docking sono solitamente costituiti da algoritmi di ricerca e funzioni di scoring. Gli algoritmi di ricerca servono per generare quante più possibili orientazioni del \textbf{ligando} (una piccola molecola organica) all’interno del \textbf{recettore} (pose), mentre spetta alla funzione di scoring classificare tali orientazioni assegnando loro un \textbf{punteggio} (score) sulla base delle interazioni ligando-proteina ad esse associate.

\section*{1.2 Idea e sviluppo}
\def\baselinestretch{1.66}
\noindent L’\textbf{idea} nasce dall’attività di tirocinio svolta presso il "Centro nazionale di ricerca" di Napoli, per un totale di 300 ore (12 CFU), sotto la supervisione del Responsabile del laboratorio professore Angelo Ciaramella. Il lavoro effettuato è consistito nella realizzazione di un software, del tutto preliminare al progetto di tesi proposto. Lo \textbf{sviluppo} è avvenuto attraverso diverse fasi nelle quali sono stati utilizzati ed implementati i seguenti tools: software per l'esecuzione del docking, funzioni di bioinformatica per la preparazione degli input necessari e software per l'analisi dei risultati dell'intero processo. Sono state determinate le componenti software ideali per automatizzare il processo di docking conseguendo risultati efficienti per quanto riguarda l'output e l'analisi dello stesso, offrendo una buona usabilità del prodotto realizzato mediante una semplice ed intuitiva interfaccia grafica.

\section*{1.3 Contenuto della tesi}
\def\baselinestretch{1.66}
\noindent La tesi è divisa in vari moduli: nella prima parte verranno discusse le tecnologie e le piattaforme scelte per la realizzazione del software, linguaggi di programmazione, strumenti di bioinformatica e piattaforme utilizzate. Nella seconda verrà illustrata l'applicazione realizzata, dalla preparazione dei ligandi e recettori, passando per il docking, finendo con l'estrazione dei legami dall'output ottenuto. Nell'ultima parte verranno tratte le conclusioni e saranno indicati gli sviluppi futuri del software realizzato.