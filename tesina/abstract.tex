\chapter*{\centering Abstract}
\begin{adjustwidth}{25pt}{25pt}
\fontsize{18pt}{14 pt}\selectfont
\textit{Le api recano importanti benefici e servizi ecologici per la società. Con l’impollinazione le api svolgono una funzione strategica per la conservazione della flora, contribuendo al miglioramento ed al mantenimento della biodiversità.\newline
La presenza di una possibile relazione tra la mortalità primaverile delle api e l’impiego di trattamenti antiparassitari in agricoltura potrebbe contribuire a comprendere meglio fenomeni complessi come la moria delle api e lo spopolamento degli alveari, che negli ultimi dieci anni hanno colpito questo settore.\newline
Lo scopo della presente tesi è quello di illustrare la progettazione di un software che visualizza come le molecole di specifici pesticidi si dispongono, in maniera spaziale, quando sono legate ai recettori delle api, questo processo viene definito docking molecolare, e successivamente il software estrae i legami che si vengono a formare.\newline
Il software realizzato prende il nome di Computational Docking, i cui obbiettivi sono: la creazione di un applicativo capace di automatizzare i processi di preparazione degli input per il docking, di esecuzione del docking e l'analisi dei suoi risultati, riunendo in una sola applicazione diverse funzioni e strumenti di bioInformatica.\newline}

\emph{"Ogni ape porta in sé il meccanismo dell’universo: ognuna riassume il segreto del mondo."}

\begin{flushright}
    \emph{Michel Onfray}
\end{flushright}

\end{adjustwidth}