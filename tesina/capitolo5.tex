\chapter{Conclusioni e sviluppi futuri}
Nell'idea iniziale del progetto \textbf{Computational Docking} gli obbiettivi erano quelli di creare un software capace di automatizzare i processi di preparazione degli input per il docking, di esecuzione del docking e l'analisi dei suoi risultati, riunendo in una sola applicazione diverse funzioni e strumenti di bioInformatica. Raggiunti gli obbiettivi di cui sopra si è deciso di dotare il software prodotto di un'interfaccia grafica, per permettere anche a chi non abbia particolari conoscenze informatiche e familiarità con l'uso della riga di comando, di utilizzare l'applicazione in modo intuitivo, inoltre sono state aggiunte ulteriori funzionalità al software quali:

\begin{itemize}
    \item casi di analisi aggiuntivi
    \item altre opzioni di esecuzione. 
\end{itemize}

Il software descritto all'interno di tale documento ha la finalità di dare un apporto importante alla ricerca scientifica, in particolare è  progettato con lo scopo di analizzare dell'effetto dei pesticidi sull'Apis mellifera. Più nello specifico l'applicazione trattata ha più scopi: 

\begin{itemize}
    \item Riunire in un unico software tutti i tools e strumenti per effettuare il docking
    \item Fornire strumenti e procedure per l'analisi dei risultati del docking
    \item Automatizzare l'intero processo di docking e di analisi
    \item Dare la possibilità di sfruttare un'applicazione semplice, che non richiede particolari conoscenze in ambito informatico per l'utente finale.
\end{itemize}

\section{Conclusioni}
\textbf{Computational Docking} automatizza l'intero processo di screening biochimico partendo dalla preparazione dei \textbf{ligandi} e dei \textbf{recettori}, passando per il processo di \textbf{docking} e concludendo con l'analisi dell'output derivato dalla fase precedente.\newline
Viene concessa ampia libertà all'utente per quanto riguarda la scelta degli input e le impostazioni d'uso del software.
Il software realizzato, come si evince dai risultati sperimentali, è stato in grado di fornire risultati soddisfacenti dal punto di vista qualitativo, poichè sia i risultati del docking che l'analisi hanno soddisfatto i requisiti e le richieste del nostro esperto del dominio, il dottor Febbraio. Da un punto di vista quantitativo sono stati provati diversi output sia per tipo che per numero e l'applicazione è sempre stata in grado produrre risultati con tempistiche equiparabili ad altri software simili (Gold, Gnina, Keplero, ecc...).\newline
Un altro aspetto di cui si è tenuto conto è stata l'usabilità dell'applicazione, infatti \textbf{Computational Docking} è dotato di una \textbf{GUI} semplice da usare e che si avvicina non solo ai modelli di applicazione maggiormente diffusi sul mercato, facilitandone la comprensione da parte dell'utente, ma che gode di una buona responsività non solo nei confronti di chi la usa, il quale è sempre tenuto al corrente di ciò che il software stia facendo, ma anche per quanto riguarda i tempi di esecuzione, infatti tramite l'uso del parallelismo e della concorrenza sviluppati attraverso i thread, è possibile effettuare diversi compiti eseguiti dal software in parallelo e con tempi di risposta contenuti. E' bene ricordare che per andare incontro a macchine le quali non dispongono di molte risorse hardware è possibile utilizzare la versione da riga di comando del software, veloce e leggera.\newline
Inoltre l'applicazione è scalabile considerata la facilità di evoluzione per sviluppi futuri.
In conclusione \textbf{Computational Docking} è in grado di eseguire il docking, preparare i suoi input ed analizzarne gli output in maniera efficiente ed efficace.

\section{Sviluppi futuri}
Il motore dello sviluppo di \textbf{Computational Docking} è stato quello di analizzare gli effetti dei composti chimici contenuti all'interno dei pesticidi più diffusi, sulle api del genere Apis mellifera. L'obbiettivo a lungo termine dell'applicazione consiste nell'ampliare il suo dominio di analisi ad altre specie e casi, per esempio l'uomo, questo è l'obbiettivo ambizioso futuro per è progettato il software.\newline
Da un punto di vista puramente pratico una delle finalità principali dell'applicazione è quella di concedere sempre più libertà per quanto riguarda la modalità di input dei dati, nel caso specifico l'obbiettivo è quello di usare come banca dati non solo i database di \textbf{Pubchem} ed \textbf{RCSB}, ma usufruire anche degli altri archivi di composti chimici ed organici presenti in internet. Per quanto riguarda l'aspetto degli input sarebbe utile trovare una modalità di input per i \textbf{ligandi} simile a quella sfruttata per i \textbf{recettori}, garantendo una maggiore quantità di input da poter fornire.\newline
Da un punto di vista computazionale si mirerà ad usare pattern, strategie, algoritmi e schemi di programmazione che mireranno a ridurre sempre di più la complessità di tempo e spazio di \textbf{Computational Docking}.\newline
Detto ciò è chiaro come questo non sia un punto di arrivo ma un solo il punto di partenza del progetto \textbf{Computational Docking}.

























































































































































































































