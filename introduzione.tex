\chapter*{\centering Introduzione}
\pagestyle{plain}
\begin{adjustwidth}{25pt}{25pt}
\fontsize{18pt}{14pt}\selectfont
\textit{Negli ultimi anni l'informatica sta ricoprendo un ruolo sempre più preponderante nel campo della "Life Science", attraverso lo sviluppo di applicazioni che permettono l’analisi o la predizione di fenomeni biologici. Partendo da un problema biologico come la crescente mortalità delle api e da uno studio biochimico e di risk assessment, che cerca di mettere in relazione questa mortalità con la predizione dell’interazione dei pesticidi con le proteine delle api, si è cercato di realizzare/progettare un software in grado di automatizzare una serie di procedure per rendere più accessibili questo tipo di analisi a ricercatori con background informatici molto diversi. Il software realizzato prende il nome di AUtomated DOcking 4 RIsk ASsessment (AUDO4RIAS), e riunisce in una sola applicazione diverse funzioni e strumenti di bioinformatica. In particolare, questo software è stato sviluppato con l’obiettivo di creare un applicativo capace di automatizzare: (i) i processi di preparazione degli input per l’analisi di docking molecolare, inclusi la ricerca e il download delle strutture 3D delle molecole (ligandi e recettori), (ii) l’esecuzione dell’analisi di docking molecolare e (iii) l’analisi dei risultati della predizione dell’interazione proteina/ligando. Il software è disponibile su GitHub, rilasciato sotto licenza Creative Commons.}
\end{adjustwidth}
\newpage 
{\fontsize{18pt}{14pt}\selectfont
\emph{\newline
\newline
\newline
\newline
\newline
\newline
\newline
\newline
\newline
\newline
\newline
\newline
"Ogni ape porta in sé il meccanismo dell’universo: ognuna riassume il segreto del mondo."}

\begin{flushright}
    \emph{Michel Onfray}
\end{flushright}}