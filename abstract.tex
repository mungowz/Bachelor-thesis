\chapter*{\centering Abstract}
\pagestyle{plain}
\begin{adjustwidth}{25pt}{25pt}
\fontsize{18pt}{14pt}\selectfont
\textit{In recent years, information technology is playing an increasingly prominent role in the field of "Life Science," through the development of applications that enable the analysis or prediction of biological phenomena. Starting from a biological problem such as increasing bee mortality and a biochemical and risk assessment study, which seeks to relate this mortality to the prediction of the interaction of pesticides with bee proteins, an attempt was made to implement/design a software capable of capable of automating a series of procedures to make this type of analysis more accessible to researchers with very different computer science backgrounds. The software created is called AUtomated DOcking 4 RIsk ASsessment (AUDO4RIAS), and it brings together in a single application different bioinformatics functions and tools. In particular, this software was developed with the goal of creating an application capable of automating: (i) the processes of preparing inputs for molecular docking analysis, including searching and download the 3D structures of molecules (ligands and receptors), (ii) the execution of the analysis of molecular docking, and (iii) the analysis of the results of the protein/ligand interaction prediction. The software is available on GitHub, released under a license of Creative Commons.}
\end{adjustwidth}