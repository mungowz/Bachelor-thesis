\chapter*{Ringraziamenti}
\markboth{RINGRAZIAMENTI}{}
Terminato il liceo non avevo la minima idea di che percorso intraprendere, le superiori non sono riuscite a farmi appassionare o propendere per qualche materia in particolare, decisi quindi di seguire il consiglio di mio padre, studiare informatica, una materia che non conoscevo minimamente ma che prometteva tanti sbocchi e grossa richiesta nel mondo del lavoro. Al primo anno di università ho scritto il mio primo programma, il classico main.c che stampava nel terminale la scritta "Hello, World!", fui fulminato, finalmente qualcosa di concreto e per cui vale la pena studiare. Da quel momento iniziò la carriera universitaria, un percorso attraverso il quale ho imparato cosa significasse responsabilità, ho ricevuto soddisfazioni e delusioni, sono cresciuto e affrontato una pandemia. Il mio mondo è cambiato, ho conosciuto persone nuove, altre se ne sono andate, ho vissuto momenti di grande felicità ma anche periodi di grande sconforto. Non so dire se nel complesso siano stati tre anni di gioia o di aver passato anni migliori, ma una cosa è certa non ero solo e qualcuno nel bene o nel male ci è sempre stato.\newline
Mia mamma Assunta, quante volte mi ha visto triste o pensieroso, cercando sempre di capire cosa stesse succedendo nonostante la scacciassi, e quante volte dopo una lite a causa del mio carattere era nuovamente lì a preoccuparsi per me, lei che più di chiunque mi capisce. Mio padre Franco che ha sempre sperato il meglio per me e mi ha sempre accompagnato in qualsiasi cosa facessi essendo orgoglioso di duo figlio, a modo suo mi ha sempre aiutato, anche fino a notte tarda. Grazie ad entrambi ed ai loro sacrifici ho avuto la possibilità di trascorrere questi tre anni senza dovermi occupare di problemi che non fossero strettamente legati al mondo universitario, ma questa è solo l'ennesima dimostrazione di quanto mi vogliano bene. La mia gattina Carry, entrata in casa all'inizio della pandemia e che ha passato ore chiusa con me in camera a riascoltare le lezioni. Un grazie a tutti i miei zii i quali mi hanno regalato sempre tanto affetto, un grazie ai miei cugini, i fratelli e le sorelle che non ho mai avuto, un grazie anche a chi è lassù e mi guarda felice oggi più che mai.\newline
Anche i miei compagni non stati da meno, in ordine casuale: Denny, Max e Dom, amici prima che colleghi, si sono sacrificati nel fare progetti con me sopportando il mio carattere difficile, spero di avervi regalato qualche sorriso durante il lavoro, IPT e Task Force sempre nel cuore. In ordine sempre casuale: Gianfranco, Gianluca, Ilaria, Peppe, Zib e Kekko, amici con cui non ho avuto il piacere di lavorare insieme in qualche progetto, ma con i quali ho vissuto momenti bellissimi dentro e fuori l'università, vi voglio bene ragazzi. Grazie  di cuore al professor Ciaramella, per la sua disponibilità e simpatia e grazie di cuore al dottor Febbraio, sempre gentile e pieno di consigli. Grazie a tutti, a chi ci è stato e mi ha voluto bene, è anche grazie a voi se oggi sono quello che sono.\newline
Con queste parole dichiaro la fine di un ciclo e l'inizio di un altro.