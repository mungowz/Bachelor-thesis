\chapter*{Conclusioni e sviluppi futuri}
\markboth{CONCLUSIONI E SVILUPPI FUTURI}{}
Nell’idea iniziale del progetto AUDO4RIAS gli obiettivi erano quelli di creare un software a riga di comando capace di automatizzare i processi di preparazione degli input per il docking, di esecuzione del docking e l’analisi dei suoi risultati, riunendo in una sola applicazione diverse funzioni e strumenti di bioinformatica. Successivamente si è deciso di dotare il software di un’interfaccia grafica, in modo da permetterne un utilizzo più intuitivo e user frendly anche a chi non abbia particolari conoscenze informatiche e familiarità con l’uso della riga di comando. Inoltre, questo ci ha permesso di aggiungere ulteriori funzionalità al software, quali:

\begin{itemize}
    \item casi di analisi aggiuntivi
    \item altre opzioni di esecuzione. 
\end{itemize}

Il software sviluppato durante il mio lavoro di tesi può avere interessanti applicazioni nel campo della ricerca scientifica, in particolare negli studi di risk assessment, infatti è stato inizialmente progettato con lo scopo di analizzare l’effetto dei pesticidi sulle proteine da Apis mellifera, tuttavia può essere impiegato in altri approcci sperimentali. Più nello specifico l’applicazione sviluppata ha più scopi: 

\begin{itemize}
    \item Riunire in un unico software tutti i tools e strumenti per effettuare un analisi di docking
    \item Fornire strumenti e procedure per l’analisi dei risultati del docking
    \item Automatizzare l’intero processo di docking e di analisi
    \item Dare la possibilità di sfruttare un’applicazione semplice, che non richiede particolari conoscenze in ambito informatico per l’utente finale.
\end{itemize}

\section*{Conclusioni}
Il software AUDO4RIAS automatizza l’intero processo di virtual screening partendo dalla creazione delle library alla preparazione delle strutture dei ligandi e dei recettori, passando per il processo di docking e concludendo con l’analisi dell’output derivato dalla fase precedente.
Inoltre, viene concessa ampia libertà all’utente per quanto riguarda la scelta degli input e le impostazioni d’uso del software, in modo da non limitare l’utente esperto. Dal punto di vista funzionale, il software è stato in grado di fornire risultati qualitativamente soddisfacenti, poichè sia i risultati del docking che dell’analisi sono stati comparabili, se non identici, a quelli ottenuti manualmente con un livello di difficoltà e con tempi, nell’ordine di giorni, significativamente superiori. Da un punto di vista quantitativo sono stati provati diversi output sia per tipo che per numero e l’applicazione è sempre stata in grado produrre risultati con tempistiche equiparabili ad altri software simili (Gold, Gnina, Keplero, ecc...). Un altro aspetto di cui si è tenuto conto è stata l’usabilità dell’applicazione. Infatti AUDO4RIAS è dotato di una GUI semplice da usare e che si avvicina non solo ai modelli di applicazione maggiormente diffusi sul mercato, ma tiene l’utente sempre al corrente di ciò che il software sta facendo, facilitandone la comprensione da parte dell’utente.Per quanto riguarda i tempi di esecuzione, il software gode di una buona responsività nei confronti di chi la usa, infatti tramite l’uso del parallelismo e della concorrenza sviluppati attraverso i thread, è possibile effettuare diversi compiti eseguiti dal software in parallelo e con tempi di risposta contenuti. Inoltre, va ricordato che, per andare incontro a macchine che non dispongono di molte risorse hardware, è sempre ossibile utilizzare la versione da riga di comando veloce e leggera del software.
Infine, l’applicazione è scalabile considerata la facilità  di evoluzione per sviluppi futuri. In conclusione, Essendo in grado di eseguire il docking, preparare i suoi input ed analizzarne gli output in maniera efficiente ed efficace, l’applicativo AUDO4RIAS si propone di essere utilizzata nel campo della ricerca e diventare oggetto di publicazioni scientifiche.

\section*{Sviluppi futuri}
Lo scopo alla base dello sviluppo di AUDO4RIAS è stato quello di realizzare un software in grado di analizzare i possibili effetti dei pesticidi più diffusi sulle proteine di Apis mellifera in maniera rapida ed automatizzata. L’obiettivo a lungo termine dell’applicazione consiste nell’ampliare il suo dominio di analisi ad altre specie e casi, come ad esempio il topo oppure l’uomo.
Da un punto di vista puramente pratico una delle finalità principali dell’applicazione è quella di concedere sempre più libertà per quanto riguarda la modalità di input dei dati, nel caso specifico l’obiettivo è quello di usare come banca dati non solo i database di Pubchem ed RCSB, ma usufruire anche degli altri archivi di composti chimici o organici e di macromolecole biologiche presenti in internet. Per quanto riguarda l’aspetto degli input sarebbe utile trovare una modalità di input per i ligandi simile a quella sfruttata per i recettori, garantendo una maggiore quantità di input da poter fornire. Da un punto di vista computazionale si mirerà ad usare pattern, strategie, algoritmi e schemi di programmazione che mireranno a ridurre sempre di più la complessità di tempo e spazio di AUDO4RIAS. Alla luce di quanto detto, risulta chiaro come questo non sia un punto di arrivo ma il punto di partenza del progetto AUDO4RIAS.